\documentclass[12pt]{article}
\usepackage[usenames]{color} %used for font color
\usepackage{amsmath, amssymb, amsthm}
\usepackage{wasysym}
\usepackage[utf8]{inputenc} %useful to type directly diacritic characters
\usepackage{graphicx}
\usepackage{caption}
\usepackage{subcaption}
\usepackage{float}
\usepackage{mathtools}
\usepackage [english]{babel}
\usepackage [autostyle, english = american]{csquotes}
\MakeOuterQuote{"}
\graphicspath{ {./} }
\newcommand{\Z}{\mathbb{Z}}
\newcommand{\N}{\mathbb{N}}
\newcommand{\R}{\mathbb{R}}
\newcommand{\Q}{\mathbb{Q}}
\newcommand{\prob}{\mathbb{P}}
\newcommand{\degrees}{^{\circ}}
\DeclarePairedDelimiter\ceil{\lceil}{\rceil}
\DeclarePairedDelimiter\floor{\lfloor}{\rfloor}

\author{Tianshuang (Ethan) Qiu}
\begin{document}
\title{Math 104, HW10}
\maketitle
\newpage

\section{Q1: Ross 33.7}
\subsection{a}
Let $P$ be an arbitrary partition such that $P=\{a=t_1 < t_2 < ... < t_n = b\}$. Now consider
$$U(f^2,P) - L(f^2,P) = (M[t_1,t_2]-m[t_1,t_2])(t_2-t_1) + ... + (M[t_{n-1},t_n]-m[t_{n-1},t_n])(t_{n}-t_{n-1})$$

Since RHS and LHS has the same partition $P$, each $t_k$ is also the same on the RHS. We consider just $P_1=[t_1, t_2]$:
\newline
Let the $M(f^2,P_1)=f(x_0)^2, m(f^2,P_1)=f(x_1)^2$
$$U(f^2,P_1) - L(f^2,P_1) = (f(x_0)^2-f(x_1)^2)(t_2-t_1) = (f(x_0)+f(x_1))(f(x_0)-f(x_1))(t_2-t_1)$$
Now consider the same partition for $f$:
$$U(f,P_1) - L(f,P_1) =(M(f,P_1)-m(f,P_1))(t_2-t_1)$$
Since $B \geq f(x)$ for all $x \in [a,b]$, we have $2B \geq (f(x_0)+f(x_1))$. Since $M(f,P_1)$ is the maximum of the function over this interval and $m(f,P_1)$ is the minimum, their difference is greater than any other differences in this interval $P_1$, namely $M(f,P_1)-m(f,P_1) \geq (f(x_0)-f(x_1))$.
\newline
Therefore we have $U(f^2,P_1) - L(f^2,P_1) \leq 2B(U(f,P_1) - L(f,P_1))$. Now we can repeat this with all intervals of $[t_k, t_k-1]$ where $2 \leq k \leq n$, thus we have shown that $$U(f^2,P) - L(f^2,P) \leq 2B(U(f,P) - L(f,P))$$ for any partition $P$

\subsection{b}
Since $f$ is integrable, for any $\epsilon > 0$, there exists a partition $P$ such that $U(f,P)-L(f,P)<\epsilon$. Now for any $\epsilon > 0$, choose $\epsilon_0 = \epsilon \times 4B$ where $B$ is the absolute bound for $f$, since $f$ is integrable we find $P_0$ that the difference between the Darboux sums is less than $\epsilon_0$
\newline
Now consider $U(f^2,P_0)-L(f^2,P_0)$, from part(a) we know that $U(f^2,P_0)-L(f^2,P_0) \leq 2B(U(f,P_0)-L(f,P_0)) \leq \frac{\epsilon}{2}<\epsilon$
\newline
Therefore $f^2$ is integrable.

\section{Q1, Ross 33.8}
By our theorem we know that the sum(difference) of two integrable functions is integrable. Therefore we know that $(f+g)$ and $(f-g)$ are integrable. By 33.7 we know that $(f+g)^2$, $(f-g)^2$ are integrable. Now we simply take the difference: $(f+g)^2-(f-g)^2 = 4fg$. We apply the integrability theorem again and we know that this is integrable as well. Thus $fg$ must be integrable/
\newpage

\section{Q2}
\subsection{a}
The function is continuous at decreasing intervals as $|x|$ decreases, it is also continuous at $x=0$. Since $-1 \leq \text{sgn}(x) \leq 1$, $-x \leq f(x)\leq x$ therefore the function converges to $0$ at $x=0$ by squeeze theorem.
\newline
Now for any other point, the continuity breaks when $\sin \frac{1}{x}$ is 0 since the whole expression which was not 0 before suddenly "drops" or "rises" to 0. More rigorously, let $\sin \frac{1}{x_0} = 0$,
then $f(x_0)=0$, within any $\delta > 0$ $\sin \frac{1}{x_0+\delta}$

\end{document}
