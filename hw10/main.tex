\documentclass[12pt]{article}
\usepackage[usenames]{color} %used for font color
\usepackage{amsmath, amssymb, amsthm}
\usepackage{wasysym}
\usepackage[utf8]{inputenc} %useful to type directly diacritic characters
\usepackage{graphicx}
\usepackage{caption}
\usepackage{subcaption}
\usepackage{float}
\usepackage{mathtools}
\usepackage [english]{babel}
\usepackage [autostyle, english = american]{csquotes}
\MakeOuterQuote{"}
\graphicspath{ {./} }
\newcommand{\Z}{\mathbb{Z}}
\newcommand{\N}{\mathbb{N}}
\newcommand{\R}{\mathbb{R}}
\newcommand{\Q}{\mathbb{Q}}
\newcommand{\prob}{\mathbb{P}}
\newcommand{\degrees}{^{\circ}}
\DeclarePairedDelimiter\ceil{\lceil}{\rceil}
\DeclarePairedDelimiter\floor{\lfloor}{\rfloor}

\author{Tianshuang (Ethan) Qiu}
\begin{document}
\title{Math 104, HW10}
\maketitle
\newpage

\section{Q1: Ross 33.7}
\subsection{a}
Let $P$ be an arbitrary partition such that $P=\{a=t_1 < t_2 < ... < t_n = b\}$. Now consider
$$U(f^2,P) - L(f^2,P) = (M[t_1,t_2]-m[t_1,t_2])(t_2-t_1) + ... + (M[t_{n-1},t_n]-m[t_{n-1},t_n])(t_{n}-t_{n-1})$$

Since RHS and LHS has the same partition $P$, each $t_k$ is also the same on the RHS. We consider just $P_1=[t_1, t_2]$:
\newline
Let the $M(f^2,P_1)=f(x_0)^2, m(f^2,P_1)=f(x_1)^2$
$$U(f^2,P_1) - L(f^2,P_1) = (f(x_0)^2-f(x_1)^2)(t_2-t_1) = (f(x_0)+f(x_1))(f(x_0)-f(x_1))(t_2-t_1)$$
Now consider the same partition for $f$:
$$U(f,P_1) - L(f,P_1) =(M(f,P_1)-m(f,P_1))(t_2-t_1)$$
Since $B \geq f(x)$ for all $x \in [a,b]$, we have $2B \geq (f(x_0)+f(x_1))$. Since $M(f,P_1)$ is the maximum of the function over this interval and $m(f,P_1)$ is the minimum, their difference is greater than any other differences in this interval $P_1$, namely $M(f,P_1)-m(f,P_1) \geq (f(x_0)-f(x_1))$.
\newline
Therefore we have $U(f^2,P_1) - L(f^2,P_1) \leq 2B(U(f,P_1) - L(f,P_1))$. Now we can repeat this with all intervals of $[t_k, t_k-1]$ where $2 \leq k \leq n$, thus we have shown that $$U(f^2,P) - L(f^2,P) \leq 2B(U(f,P) - L(f,P))$$ for any partition $P$

\subsection{b}
Since $f$ is integrable, for any $\epsilon > 0$, there exists a partition $P$ such that $U(f,P)-L(f,P)<\epsilon$. Now for any $\epsilon > 0$, choose $\epsilon_0 = \epsilon \times 4B$ where $B$ is the absolute bound for $f$, since $f$ is integrable we find $P_0$ that the difference between the Darboux sums is less than $\epsilon_0$
\newline
Now consider $U(f^2,P_0)-L(f^2,P_0)$, from part(a) we know that $U(f^2,P_0)-L(f^2,P_0) \leq 2B(U(f,P_0)-L(f,P_0)) \leq \frac{\epsilon}{2}<\epsilon$
\newline
Therefore $f^2$ is integrable.

\section{Q1, Ross 33.8}
By our theorem we know that the sum(difference) of two integrable functions is integrable. Therefore we know that $(f+g)$ and $(f-g)$ are integrable. By 33.7 we know that $(f+g)^2$, $(f-g)^2$ are integrable. Now we simply take the difference: $(f+g)^2-(f-g)^2 = 4fg$. We apply the integrability theorem again and we know that this is integrable as well. Thus $fg$ must be integrable.
\newpage

\section{Q2}
\subsection{a}
The function is continuous at decreasing intervals as $|x|$ decreases, it is also continuous at $x=0$. Since $-1 \leq \text{sgn}(x) \leq 1$, $-x \leq f(x)\leq x$ therefore the function converges to $0$ at $x=0$ by squeeze theorem.
\newline
Now for any other point, the continuity breaks when $\sin \frac{1}{x}$ is 0 since the whole expression which was not 0 before suddenly "drops" or "rises" to 0. More rigorously, if $\sin \frac{1}{x} \not = 0$, then sgn$(x)=1$ or $-1$, then $f(x)=x$ or $-x$, which does not have the value $0$ unless $x=0$. Therefore the function is discontinuous at all points where $\sin(\frac{1}{x})=0$, or $\frac{1}{x} = n \pi$ where $n$ is an integer.
Since it is continuous everywhere else, we have that the function is continuous on $[-1,-\frac{1}{\pi}), (-\frac{1}{\pi},-\frac{1}{2\pi})... (\frac{1}{2\pi},\frac{1}{\pi}), (\frac{1}{\pi},1]$

\subsection{b}
Even though $f$ is not piecewise continuous on all of $[-1,1]$, the discontinuities increase near 0. We claim that it is piecewise continuous on $[-1,0)$ and $(0,1]$. Let $a_n$ be the sequence of postive discontinuous points: $a_n = \frac{1}{n\pi}$, and let $b_n=-a_n$. Since $0<a_n < 1/n$ we know that it converges to $0$, by similar logic so does $b_n$.
Let $0<x_0\leq 1$, we know that between each $a_n$ the function is either $x$ or $-x$ which is uniformly continuous. Moreover, we know that since $a_n \to 0$, there is $n \in \N$ such that $a_n < x_0$. Let $n_0$ be the smallest such $n$. Let the first partition be $[x_0, a_{n_0-1}]$, and the last $[a_1,1]$.
The same is true if $-1 \leq x_1 < 0$, let $n_1$ be the smallest $n \in \N$ such that $a_{n} > x_1$. We define our first partiton as $[-1,a_1]$, and the last $[a_{n},x_1]$, between each closed interval the function is uniformly continuous.
\newline
Now for any $\epsilon > 0$, choose $u = \sqrt{\epsilon}/4, v = - u$. As we have shown above the function is integrable in $[-1,v]$ and $[u,1]$. Therefore we only need to consider the interval $[v,u]$. In this interval the greatest value $f$ can take is $u$ when $\text{sgn}(\sin(\frac{1}{x}))=1$ and $f(x)=x$
Similarly the least value it can take is $v$ when $f(x)=-x$
$$U(f,[v,u]) = u(u-v)$$
$$L(f, [v,u]) = v(u-v)$$
Thus $U(f,[v,u])-L(f, [v,u]) = (u-v)^2 = \frac{\epsilon}{16} < \epsilon$. Therefore the function is integrable.
\newpage


\section{Ross 34.2}
\subsection{a}
We first assumes that the function $e^{t^2}$ is the derivative of a function $F(t)$, so by the Fundamental Theorem of Calculus, we simplify the expression into $\lim_{x \to 0} \frac{F(x)-F(0)}{x}$.
Since the denominator approaches 0, we apply L'Hospital's rule and the limit is equal to $\lim_{x \to 0} \frac{F'(x)-F'(0)}{1} = \frac{e^{x^2}(x')-e^0(0')}{1} = 1$
\newline
In the last step, we needed to apply the chain rule and take the derivative of the function inputs, ending with $1-0=1$

\subsection{b}
We first assumes that the function $e^{t^2}$ is the derivative of a function $F(t)$, so by the Fundamental Theorem of Calculus, we simplify the expression into $\lim_{x \to 0} \frac{F(3+h)-F(3)}{h}$.
Since the denominator approaches 0, we apply L'Hospital's rule and the limit is equal to $\lim_{x \to 0} \frac{F'(3+h)-F'(3)}{1} = \frac{e^{(3+h)^2}(3+h)'-e^3(0')}{1} = e^9$
\newpage


\section{Q4}
\subsection{a}
For $x < 0$, $F(x)=0$ since $f(x)=0$
\newline
For $0 \leq x \leq 1$, $F(x)=\frac{1}{2}x^2$ since we have proven the power rule and $\frac{1}{x}x^2$ has a derivative of
$x$.
\newline
For $x > 1$, $F(x) = \frac{1}{2} + 4(x-1)$. The one half comes from $F(1)-F(0)$, and since the function is constant, the upper and lower Darboux sums will be the same for any partition.

\subsection{b}
$F$ is continuous everywhere. In the intervals $x<0, 0<x<1, x>1$ we know $F$ is continuous since their functions are continuous. At $x=0$, let $\epsilon_0 > 0$, pick $\delta_0 = \sqrt{\epsilon_0}$
Let $|x-0|<\delta$, then if $x<0$, $|F(x)-F(0)|=0$, otherwise $|F(x)-F(0)| < \epsilon_0/2-0 < \epsilon$, therefore $F$ is continuous at 0
\newline
At $x=1$, let $\epsilon_1 > 0$, pick $\delta_1 = \sqrt{\epsilon_0}$. If $x>1$

\subsection{c}
$F$ is differentiable at $(-\infty, 1)$. $F'(x)$ for $x < 0$ is $0$ since it is constant. $F'(x)$ on $[0,1]$ is $x$ and since $F'(x)=0$ on both negative and positive sides of $x=0$, therefore $F$ is differentiable at $0$
\newline
For $x>1$, $F'(x)= 4$. Therefore it is also differentiable at $(1, \infty)$. It is not differentiable at $1$ since on the negative side it is $1$, but on the positive it is $4$
\newpage


\section{Ross 34.5}
For each $x \in \R$, limit both $F,f$ to $[x-1,x+2]$. Since $f$ is continuous on this interval and $F$ is its integral, $F$ is differentiable at $x$ by the Fundamental Theorem of Calculus.
\newline
Since the upper bound of the integral computed in $F$ has an upper bound of $x+1$, by the same theorem we know that it is equal to $f(x+1)$ .
\newpage


\section{Ross 34.7}
Let $J$ be the interval $(-\infty, \infty)$, $u:J \to \R $ is defined as $u=x^2$ and $u' = 2x$, and $f: \R \to \R$, $f = -\frac{2}{3}(1-a)^{3/2}$, by the power rule $f' = \sqrt{1-a}$. Now since $U(J) \subseteq \R$, we can apply u-subsitution here.
\newline
Let the integral equal $I$, we know that $2I = \int_0^1 2x \sqrt{1-x^2} dx= \int_0^1 u'(f'\circ u) = \int_{u(0)}^{u(1)}f' = f(1)-f(0) = \frac{2}{3}$
\newline
Now we have $2I = \frac{2}{3}$, so our original integral $I = \frac{1}{3}$
\newpage

\section{Ross 34.12}
Assume that there exists some $f$ where it is not $0$ everywhere that satisfies this integrala. Since $g$ is an arbitrary continuous function and $f$ is continuous, let $g=f$. Now our integral becomes $\int_a^b f^2(x)dx = 0$
Now since $f^2(x) \geq 0$, we know that $f^2(x) = 0$ everywhere in our interval.
\newline
To see that it is true, we let $f^2(x_0) > 0$, then by continuity there exists $\delta > 0$ such that $|f^2(x)-f^2(x_0)|< f^2(x_0)/2$, then in this interval the integral is at least $\delta f^2(x_0)/2$, thus the whole integral must be greater than or equal to it. Since the expression is positive, there is no way for the integral to be 0. Thus $x_0$ does not exist and $f^2(x)=0$ in our interval.
\newline
From our conclusion above it is simple to see that $f(x)=0$ everywhere for $f^2(x)=0$, therefore we have reached a contradiction, and $f(x)=0$ for all $x \in [a,b]$
\end{document}
