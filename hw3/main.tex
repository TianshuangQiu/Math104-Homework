\documentclass[12pt]{article}
\usepackage[usenames]{color} %used for font color
\usepackage{amssymb} %maths
\usepackage{amsmath} %maths
\usepackage[utf8]{inputenc} %useful to type directly diacritic characters
\usepackage{graphicx}
\usepackage [english]{babel}
\usepackage [autostyle, english = american]{csquotes}
\MakeOuterQuote{"}
\graphicspath{ {./} }
\newcommand{\Z}{\mathbb{Z}}
\newcommand{\N}{\mathbb{N}}
\newcommand{\R}{\mathbb{R}}
\newcommand{\Q}{\mathbb{Q}}
\newcommand{\prob}{\mathbb{P}}

\author{Tianshuang (Ethan) Qiu}
\begin{document}
\title{Homework 3}
\maketitle
\newpage

\section{Ross 8.2}

\subsection{a}
When $n>0$, both the numerator and the denomator is greater than 0, so $a_n > 0 \forall n>0$. Now consider $x_n = \frac{n}{n^2} = \frac{1}{n}$, the denomator is less than $a_n$, so $x_n > a_n \forall n>0$.
\newline
Obviously 0 converges to 0, and we have proven in class that $\frac{1}{n}$ converges to 0 (pick $N = \frac{1}{\epsilon}$). $a_n \to 0$ by squeeze theorem.


\subsection{c}
This sequence converges to $\frac{4}{7}$. We will try to show that $x_n = |c_n-\frac{4}{7}| \to 0$.
\newline
$$|c_n-\frac{4}{7}| = |\frac{28n+21}{49n-35} - \frac{4(7n-5)}{49n-35}|$$
$$= \frac{28n+21-28n+20}{49n-35}$$
$$= \frac{41}{49n-35}$$
$$\lim x_n = 41 \times \lim (\frac{1}{49n-35})$$
Once again, we squeeze this sequence with 0 and $\frac{1}{n}$. Since the denomator is positive $x_n > 0 \forall n>0$, and $49n-35 > n \forall n>\frac{35}{48}$, so $x_n < \frac{1}{n}$ when $n$ is large. Therefore $\frac{1}{49n-35} \to 0$ by squeeze theorem.
\newline
Therefore $\lim x_n = 49 \times 0 = 0$, and $c_n \to \frac{4}{7}$


\subsection{e}
The sequence converges to 0. We define $x_n = \frac{1}{n}, y_n = \frac{-1}{n}$. Since $-1 \leq \sin n \leq 1$, $y_n \leq s_n \leq x_n$.
\newline
We have shown above that $\frac{1}{n} \to 0$. $\frac{-1}{n}$ can be shown with the same reasoning to converge $\forall n>\frac{1}{\epsilon}$. Therefore $s_n \to 0$ by squeeze theorem.
\newpage


\section{Q2}
\subsection{a}
We will prove this via induction.
\newline
Base case: n = 1, $LHS = 1+a; RHS = \frac{1-a^{2}}{1-a}$. $1-a^2 = (1+a)(1-a)$, therefore LHS = RHS. Base case holds.
\newline
Inductive hypothesis, let the claim hold for an arbitrary $n \in \N$.
$$1+a+...+a^n+a^{n+1} = \frac{1-a^{n+1}}{1-a}+a^{n+1}$$
$$=\frac{1-a^{n+1}+a^{n+1}-a^{n+2}}{1-a}$$
$$=\frac{1-a^{n+2}}{1-a}$$
Thus we have proven the inductive hypothesis. Q.E.D.

\subsection{b}
Since $|s_{n+1}-s_n|<1/2^n \forall n$, $|s_{n+2}-s_n| \leq 1/2^n + 1/2^{n+1}$ by the triangle inequality. We can further extend this so $|s_{n+k}-s_n| \leq 1/2^n + 1/2^{n+1} + ... +1/2^{n+k-1}$
\newline
Let $\epsilon > 0$, we pick $N = \max \{ \log_2 (1/\epsilon), 1\}$. Pick $j,k > N$.
$$|s_j - s_k| \leq \sum ^{j+k-1}_{i=j} \frac{1}{2^i} = \frac{1}{2^j}(1+1/2+...+1/2^{k-1})$$
$$=\frac{1}{2^j}(2(1-1/2^{k})) = \frac{1-1/2^k}{2^{j-1}} \leq \frac{1}{2^j}$$
Since $j> \log_2 (1/\epsilon$), $|s_j - s_k|\leq \frac{1}{2^j} < \epsilon$.
\newline
Therefore the sequence is cauchy. Q.E.D.
\newpage


\section{Ross 10.7}
We essentially want to slowly pick items in $S$ that is closer to $\sup S$.
\newline
Consider $x_n = \sup S - 1/n$. Since $x_n < \sup S, \exists s_n \in S s.t. x_n \leq s_n \leq \sup S$. Both $x_n$ and $\sup S$ converge to $\sup S$, therefore $s_n \to \sup S$ by squeeze theorem.
\newpage


\section{Q4}



\end{document}
