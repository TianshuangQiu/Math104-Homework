\documentclass[12pt]{article}
\usepackage[usenames]{color} %used for font color
\usepackage{amsmath, amssymb, amsthm}
\usepackage{wasysym}
\usepackage[utf8]{inputenc} %useful to type directly diacritic characters
\usepackage{graphicx}
\usepackage{caption}
\usepackage{subcaption}
\usepackage{float}
\usepackage{mathtools}
\usepackage [english]{babel}
\usepackage [autostyle, english = american]{csquotes}
\MakeOuterQuote{"}
\graphicspath{ {./} }
\newcommand{\Z}{\mathbb{Z}}
\newcommand{\N}{\mathbb{N}}
\newcommand{\R}{\mathbb{R}}
\newcommand{\Q}{\mathbb{Q}}
\newcommand{\prob}{\mathbb{P}}
\newcommand{\degrees}{^{\circ}}
\DeclarePairedDelimiter\ceil{\lceil}{\rceil}
\DeclarePairedDelimiter\floor{\lfloor}{\rfloor}

\author{Tianshuang (Ethan) Qiu}
\begin{document}
\title{Math 104, HW6}
\maketitle
\newpage


\section{Q1}
Let $x_0 \in \R, \epsilon > 0$. We first define $\log_a a^k = k$. Choose $\delta = \log_a(\frac{\epsilon}{a^{x_0}}+1)$
\newline
Let $|x-x_0|<\delta$, consider $|f(x)-f(x_0)|=|a^x-a^{x_0}| \leq |a^{|x|}-a^{|x_0|}|$. Since we know that $|x-x_0|<\delta$, we have $|x|<\delta + |x_0|$
\newline
Now we subsitute our inequality in:
$$|a^{|x|}-a^{|x_0|}| < |a^{\delta + |x_0|}-a^{|x_0|}| = |a^{x_0}(a^\delta - 1)| = |a^{x_0}( \frac{\epsilon}{a^{x_0}}+1-1)| = |a^{x_0}\frac{\epsilon}{a^{x_0}}|=\epsilon$$
Therefore we have shown the epsilon delta property. $\blacksquare$
\newpage


\section{Ross 17.12}

\subsection{a}
Assume that there is a point $x_0 \in (a,b)$ where $f(x_0) \not= 0$. We now try to show that this contradicts the epsilon-delta property of continuity.
\newline
Now let $f(x_0) = k, \epsilon = |k|/2$. By continuity there exists $\delta > 0$ such that $|x-x_0|<\delta \implies |f(x)-f(x_0)|<\delta$. Now let this arbitrary delta be $\delta_0$. By the density of rationals we know that there exists $q \in \Q s.t. x_0-\delta < q < x_0+\delta$. By the definition of this function $f(q)=0$, and $|f(x_0)-f(q)|=k > \epsilon$. \lightning
\newline
We have found a contradiction of the continuity principle, so our assumption must be wrong and $f(x) = 0$ in the domain.

\subsection{b}
We can simply repeat the argument above but replace $0$ with $g(r)$. Since we never really used the value $0$ , we can safely do this.
\newline
Assume that there is a point $x_0 \in (a,b)$ where $f(x_0) \not= g(x_0)$. We now try to show that this contradicts the epsilon-delta property of continuity.
\newline
Now let $f(x_0) = k, \epsilon = |k-g(x_0)|/2$. By continuity there exists $\delta > 0$ such that $|x-x_0|<\delta \implies |f(x)-f(x_0)|<\delta$. Now let this arbitrary delta be $\delta_0$. By the density of rationals we know that there exists $q \in \Q s.t. x_0-\delta < q < x_0+\delta$. By the definition of this function $f(q)=g(q)$, and $|f(x_0)-f(q)|=|k-g(x_0)| > \epsilon$. \lightning
\newline
We have found a contradiction of the continuity principle, so our assumption must be wrong and $f(x) = 0$ in the domain.


\end{document}
