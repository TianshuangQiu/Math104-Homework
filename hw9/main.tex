\documentclass[12pt]{article}
\usepackage[usenames]{color} %used for font color
\usepackage{amsmath, amssymb, amsthm}
\usepackage{wasysym}
\usepackage[utf8]{inputenc} %useful to type directly diacritic characters
\usepackage{graphicx}
\usepackage{caption}
\usepackage{subcaption}
\usepackage{float}
\usepackage{mathtools}
\usepackage [english]{babel}
\usepackage [autostyle, english = american]{csquotes}
\MakeOuterQuote{"}
\graphicspath{ {./} }
\newcommand{\Z}{\mathbb{Z}}
\newcommand{\N}{\mathbb{N}}
\newcommand{\R}{\mathbb{R}}
\newcommand{\Q}{\mathbb{Q}}
\newcommand{\prob}{\mathbb{P}}
\newcommand{\degrees}{^{\circ}}
\DeclarePairedDelimiter\ceil{\lceil}{\rceil}
\DeclarePairedDelimiter\floor{\lfloor}{\rfloor}

\author{Tianshuang (Ethan) Qiu}
\begin{document}
\title{Math 104, HW9}
\maketitle
\newpage

\section{Q1}
\subsection{a}
To show that the inverse $g$ as defined in the problem is a function, we simply need to show that each unique input has a single output. This implies that our function $f$ must be injective on to $\R$, which is to say $f(a) = f(b) \iff a = b$.
\newline
We can assume that there exists $x_0, x_1 \in I$ such that $f(x_0) = f(x_1)$.
Then by the Mean Value Theorem we know that there is a point $y \in (x_0, x_1)$ where $f'(y) = \frac{f(x_0)-f(x_1)}{x_0-x_1}=0$
However the problem specifies that $f'(x) \not= 0$, therefore our assumption is incorrect and $f$ must be injective to $\R$. Thus its inverse $g$ exists.

\subsection{b}
We claim that $f$ is monotone. Since $f'(a) = \lim_{x \to a}\frac{f(x)-f(a)}{x-a}$ exists, we know that it satisfies the epsilon-delta property. Since $f$ is differentiable, its derivative is defined on all $I$. Therefore each point in $I$ also must satisfy the epsilon-delta property, and $f'$ is therefore continuous. Now since it is continuous, if $f'(b)>0$ and $f'(c)<0$ for some $b,c \in I$, then by Intermediate Value THeorem there must be $d \in (b,c)$ such that $f(d)=0$. However we know that to be false, therefore $f$ is monotone.
\newline
Let $\epsilon > 0, y_0 \in f(I), x_0 \in \R$ such that $f(x_0)=y_0$. Without loss of generality assume that $f$ is monotonically increasing. Therefore $f(x_0-\epsilon) < f(x_0) < f(x_0+\epsilon)$.
Now we can simply take $\delta = \min\{f(x_0)-f(x_0-\epsilon), f(x_0+s\epsilon) + f(x_0)\}$. For any $|y_1-y_0|<\delta$, let $f(x_1)=y_1$, then $x_0-\epsilon < x_1 < x_0 + \epsilon$ by monoticity. Thus $g$ is continuous.
\newline
If $f$ is monotonically decreasing we can simply repeat the above argument but with the signs flipped since if $a > b$, now we will have $f(b) > f(a)$
\newpage


\section{Ross 30.2}
\subsection{a}
$\sin(0)-0=0$, so we attempt to use l'hospital's rule. Assume that the limit exists, then it must be equal to
$$\lim_{x \to 0}\frac{x^2}{\cos x} = \lim_{x \to 0}\frac{0}{1} = 0$$
Therefore the limit is 0

\subsection{b}
For this problem we need to use l'hoospital's rule 3 times
$$\lim_{x \to 0} \frac{\tan x-x}{x^3} = \lim_{x \to 0} \frac{\sec^2 x-1}{3x^2} = \lim_{x \to 0}\frac{2\tan(x)\sec^2(x)}{6x}$$
$$= \lim_{x \to 0}\frac{2 \sec^2x (\sec^2(x) + 2 \tan^2(x))}{6} = \frac{2}{6} = \frac{1}{3}$$
We used the chain rule for the second step, both the chain rule and the product rule for the third step.

\subsection{c}
We combine these fractions, then apply l'hospital's rule twice:
$$\lim_{x\to 0}\frac{x-\sin x}{x\sin x} = \lim_{x\to 0}\frac{1-\cos x}{\sin x + x\cos x} = \lim_{x\to 0}\frac{\sin x}{\cos x + \sin x - x \sin x} $$
$$= \lim_{x\to 0}\frac{0}{1} = 0$$

\subsection{d}
We know that if $\lim_{x \to a} f(x)=b, \lim_{x \to b} g(x)=c$, then $\lim_{x \to a}g(f(x)) = g(\lim_{x \to a}f(x))$. Assume that the limit does exist for our expression, and since the natural log is continuous, we can apply this theorem.
\newline
$$\ln(\lim_{x \to 0} \cos x ^{1/x^2}) = \lim_{x \to 0} \ln(\cos x^{1/x^2}) = \lim_{x \to 0}\frac{\ln(\cos x)}{x^2}$$
Now we can use l'hospital's Theorem
$$=\lim_{x \to 0}\frac{-\sin x / \cos x}{2x} = \lim_{x \to 0}\frac{- \sec^2 x}{2} = -\frac{1}{2}$$
Now to find $\lim_{x \to 0} \cos x ^{1/x^2}$, we simply apply the inverse of the natural log:
$$e^\frac{-1}{2} = \frac{1}{\sqrt e}$$
\newpage


\section{Q3}
\subsection{a}
Since $x_n \to \infty$, $x_n$ get can arbitrarily large. More rigorously, for any $r \in \R$, there exists $n \in \N$ such that if $m > n$, $x_m > r$
\newline
Consider $y_n = \frac{1}{x_n}$. Let $\epsilon > 0$, let $\epsilon_0 = \max \{ \frac{1}{\epsilon}, 1\}$. Find $n \in \N$ such that $x_n > \epsilon 0$, which we know exists as we have shown above.
Now since $\epsilon_0 > 0$, we know that $x_n, y_n$ are positive, so we have $|y_n| = |\frac{1}{x_n}| < |\frac{1}{\epsilon_0}| \leq \epsilon$
\newline
Thus we have shown that $|y_n|$ can get arbitrarily small, therefore $y_n \to 0$
\newline
$\blacksquare$

\subsection{b}
Since $\ln_{x \to a} f(x) = \infty$, then for any $r \in \R$, there exists a $\delta > 0$ such that $|x-a|<\delta \implies f(x)>r$
\newline
Let $g(x)=\frac{1}{f(x)}$. We know that $g$ is well defined since $f(x)\not = 0$ for $x \in (a,b)$. Let $\epsilon > 0$, take $\epsilon_0 = \max \{ \frac{1}{\epsilon}, 1\}$. Find $\delta > 0$ such that $f(a+\delta) > \epsilon_0$, which we know exists as we have shown above.
\newline
$|g(a+\delta)| < \frac{1}{\epsilon_0} \leq \epsilon$ We have shown thata $|g(x)|$ gets arbitrarily small when $x$ is close to $a$, therefore $\lim_{x\to a}\frac{1}{f(x)}=0$
\newline
$\blacksquare$
\newpage


\section{Q4}
Let $P$ be a partition such that $P=\{t_1 = a < t_2 < ... < t_n = b \}$, furthermore let $P$ be evenly spaced such that $t_i-t_{i-1}$ is equal for all $1<i<n$. Let $M(s)$ denote the supremum of $f$ in a set $s$, and $m(s)$ the infimum.
\newline
We find the upper and lower Darboux Sum. Since $f(x)=x$, if $x_0 > x_1, f(x_0)>f(x_1)$, so the infimum is at the lower bound of the interval and the supremum the upper bound.
$$U(f, P) = \sum_{i=1}^n M(s)(t_i-t_{i-1}) = \sum_{i=1}^n (t_i)(t_i-t_{i-1})$$
$$L(f, P) = \sum_{i=1}^n m(s)(t_i-t_{i-1}) =  \sum_{i=1}^n (t_{i-1})(t_i-t_{i-1})$$
Since $P$ is evenly spaced, we know that $t_i = a + (b-a)i/n$
\newline
Let $\epsilon > 0$, consider $U(f,P)-L(f,P)$, we can combine the sums to get
$$U(f,P)-L(f,P) = \sum_{i=1}^n (t_i-t_{i-1})(t_i-t_{i-1}) = \frac{(b-a)^2}{n}$$
Now for any $\epsilon > 0$ we simply need to choose $n > (b-a)^2/\epsilon$, which implies that $U(f,P)-L(f,P) < \epsilon$. Since it is less than any positive number and $U(f,P) \geq L(f,P)$, we have $U(f) = L(f)$ and the function is integrable.
\newline
In order to find U(f) we need to subsitute $U(f, P) = \sum_{i=1}^n (t_{i})(t_i-t_{i-1}) = \sum_{i=1}^n (a + (b-a)i/n)((b-a)/n) = (b-a)/n(\frac{((a+ (b-a)/n) + b)n}{2})$
\newline
The last step is using the sum of an arithmetic sequence, now we can tidy up to see that $U(f, P) = \frac{(b-a)(a+(b-a)/n) + b}{2}$. Since $U(f)$ is the infimum of the set $U(f, P)$ where $P$ is a partition, and we know that this sum is minimized as $n \to \infty$ since the term $(b-a)/n \to 0$ as proven in the last problem.
Thus we have found
$$U(f) = \lim_{n \to \infty}  \frac{(b-a)(a+(b-a)/n) + b}{2} = \frac{(b-a)(b+a)}{2}$$
\newpage


\section{Ross 32.6}
Since $f$ is bounded, we know that $U_n, L_n$ are finite. Therefore we have $\lim U_n - \lim L_n = 0$. Since we know that $U(f,P)$ where $P$ is a partition has an infimum at $U(f)$ and similarly $L(f,P)$ has a supremum at $L(f)$.
We have $\lim U_n = \lim L_n$, and that $U_n \geq L_n$ with the equivalence happening if and only if they are the infimum and the supremums and the function is integrable.
We must have $\lim L_n = L(f)$ and $\lim U_n = U(f)$. Since $L(f)=U(f)$, the function is integrable. Thus we have
$$\int_a^n f = L(f)=U(f) = \lim L_n = \lim U_n$$
\newpage


\subsection{Q6}
We essentially have to show that since this is a finite subset, we can create finitely many "thin rectangles" around each of these points. These rectangles have negligible width and converges to 0.
\newline
For any $s_n \in S$, for ease of notation let $f(s_n) = y_n$. Now we consider the upper and lower Darboux sums of this function.
\newline
We define our partition $P$ as such: $P = \{a, s_n-\delta, s_n+\delta, b \}$ for each $s_n \in S$ with $\delta>0$(if $a \in S$ or $b \in S$,
our first partitions become $\{a, a+\delta, ...\}$ or our last partitions become $\{..., b-\delta, b\}$).
Since $S$ is finite, we can iterate through all the points in it with the above expression.
\newline
For the lower sum, since each partiton contains some points that are not in $S$, the infimum for every partiton is $0$, and we have $U(f,P) = 0$
\newline
For the upper sum, each partition that doesn't contain a point in $S$ has a supremum of $0$, a partiton $[s_n-\delta, s_n+\delta]$ has a supremum at $s_n$ with a value of $y_n$. So we have
$$U(f,P) \leq 2\delta \sum_{i=1}^n y_i$$
If $a$ or $b$ is in $S$, the above inequality still holds since our interval would only be $\delta$ instead of $2\delta$ and $\delta < 2\delta$
\newline
Let $z = \max \{y_1, y_1, ..., y_n\}$. For any $\epsilon >0$, choose $\delta < \frac{z}{2n}$. Now consider our inequality:
$$U(f,P) \leq 2\delta \sum_{i=1}^n y_i \leq 2\delta z < \epsilon$$
Thus we have shown that $U(f,P)$ can be less than any positive number, and we also know that $U(f,P) \geq L(f,P)=0$ so we have $U(f,P)=L(f,P)=0$. Thus the function is integrable.
\newline
Since this equivalency can only happen if $U(f,P)=U(f)$ and $L(f,P)=L(f)$, we have $\int_a ^b f = 0$
\newpage


\section{Q7}
Let $P$ be an equi-distant partition of $[0, \pi/2]$. We compute the upper and lower Darboux sums:
$$U(f, P) = \sum_{i=1}^n M(s)(t_i-t_{i-1})$$
$$L(f, P) = \sum_{i=1}^n m(s)(t_i-t_{i-1})$$
Since we know that $\sin x$ is monotonically increasing in our domain, so $M([a,b]) = \sin(b), m([a,b]) = \sin(a)$.
\newline
Since $P$ is equi-distant, we have $t_i = a + (b-a)i/n$, so when we take $U(f,P)-L(f,P)$, we can get a clean telescopic series:
$$U(f,P)-L(f,P) = \sum_{i=1}^n \sin(a + (b-a)\frac{1}{n})(b-a)/n - \sin(a + (b-a)\frac{i-1}{n})(b-a)/n)$$
$$= \frac{b-a}{n}(\sin b - \sin a)$$
We know that $a=0, b = \pi/2$, so we have $U(f,P)-L(f,P) = \frac{\pi}{2n}1$
\newline
For any $\epsilon > 0$, pick $n > \max \{\frac{\pi}{2\epsilon}\, 1\}$. After partitioning $[0, \pi/2]$ into $n$ equal partitions, the difference between our upper and lower sum is $\frac{\pi}{2n} < \frac{\pi}{\epsilon \pi} = \epsilon$
Thus we have shown that the difference between $U(f,P)$ and $L(f,P)$ can get arbitrarily small. Therefore the two converges to the same number, and $\sin x$ is integrable on our interval.
\end{document}
