\documentclass[12pt]{article}
\usepackage[usenames]{color} %used for font color
\usepackage{amsmath, amssymb, amsthm}
\usepackage{wasysym}
\usepackage[utf8]{inputenc} %useful to type directly diacritic characters
\usepackage{graphicx}
\usepackage{caption}
\usepackage{subcaption}
\usepackage{float}
\usepackage{mathtools}
\usepackage [english]{babel}
\usepackage [autostyle, english = american]{csquotes}
\MakeOuterQuote{"}
\graphicspath{ {./} }
\newcommand{\Z}{\mathbb{Z}}
\newcommand{\N}{\mathbb{N}}
\newcommand{\R}{\mathbb{R}}
\newcommand{\Q}{\mathbb{Q}}
\newcommand{\prob}{\mathbb{P}}
\newcommand{\degrees}{^{\circ}}
\DeclarePairedDelimiter\ceil{\lceil}{\rceil}
\DeclarePairedDelimiter\floor{\lfloor}{\rfloor}

\author{Tianshuang (Ethan) Qiu}
\begin{document}
\title{Math 104, HW9}
\maketitle
\newpage

\section{Q1}
\subsection{a}
To show that the inverse $g$ as defined in the problem is a function, we simply need to show that each unique input has a single output. This implies that our function $f$ must be injective on to $\R$, which is to say $f(a) = f(b) \iff a = b$.
\newline
We can assume that there exists $x_0, x_1 \in I$ such that $f(x_0) = f(x_1)$.
Then by the Intermediate Value Theorem we know that there is a point $y \in (x_0, x_1)$ where $f'(y) = \frac{f(x_0)-f(x_1)}{x_0-x_1}=0$
However the problem specifies that $f'(x) \not= 0$, therefore our assumption is incorrect and $f$ must be injective to $\R$. Thus its inverse $g$ exists.

\subsection{b}
We can use the fact that $f$ is differentiable and therefore continuous to state that it must map points that are close to each other to neighborhoods that are close to each other.
\newline
Assume that $g$ is not continuous, then there exists some $\epsilon > 0, y_0 \in f(I), f(x_0)=y_0$ such that for any $\delta > 0$, $|y-y_0|<\delta$ but $|x-x_0|>\epsilon$.
We now unpack the statement: since we are taking $y=f(x)$ in this case, $|y-y_0|<\delta$ means that $|f(x)-f(x_0)|>\epsilon$.
\newline
However




\end{document}
