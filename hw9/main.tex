\documentclass[12pt]{article}
\usepackage[usenames]{color} %used for font color
\usepackage{amsmath, amssymb, amsthm}
\usepackage{wasysym}
\usepackage[utf8]{inputenc} %useful to type directly diacritic characters
\usepackage{graphicx}
\usepackage{caption}
\usepackage{subcaption}
\usepackage{float}
\usepackage{mathtools}
\usepackage [english]{babel}
\usepackage [autostyle, english = american]{csquotes}
\MakeOuterQuote{"}
\graphicspath{ {./} }
\newcommand{\Z}{\mathbb{Z}}
\newcommand{\N}{\mathbb{N}}
\newcommand{\R}{\mathbb{R}}
\newcommand{\Q}{\mathbb{Q}}
\newcommand{\prob}{\mathbb{P}}
\newcommand{\degrees}{^{\circ}}
\DeclarePairedDelimiter\ceil{\lceil}{\rceil}
\DeclarePairedDelimiter\floor{\lfloor}{\rfloor}

\author{Tianshuang (Ethan) Qiu}
\begin{document}
\title{Math 104, HW9}
\maketitle
\newpage

\section{Q1}
\subsection{a}
To show that the inverse $g$ as defined in the problem is a function, we simply need to show that each unique input has a single output. This implies that our function $f$ must be injective on to $\R$, which is to say $f(a) = f(b) \iff a = b$.
\newline
We can assume that there exists $x_0, x_1 \in I$ such that $f(x_0) = f(x_1)$.
Then by the Mean Value Theorem we know that there is a point $y \in (x_0, x_1)$ where $f'(y) = \frac{f(x_0)-f(x_1)}{x_0-x_1}=0$
However the problem specifies that $f'(x) \not= 0$, therefore our assumption is incorrect and $f$ must be injective to $\R$. Thus its inverse $g$ exists.

\subsection{b}
We claim that $f$ is monotone. Since $f'(a) = \lim_{x \to a}\frac{f(x)-f(a)}{x-a}$ exists, we know that it satisfies the epsilon-delta property. Since $f$ is differentiable, its derivative is defined on all $I$. Therefore each point in $I$ also must satisfy the epsilon-delta property, and $f'$ is therefore continuous. Now since it is continuous, if $f'(b)>0$ and $f'(c)<0$ for some $b,c \in I$, then by Intermediate Value THeorem there must be $d \in (b,c)$ such that $f(d)=0$. However we know that to be false, therefore $f$ is monotone.
\newline
Let $\epsilon > 0, y_0 \in f(I), x_0 \in \R$ such that $f(x_0)=y_0$. Without loss of generality assume that $f$ is monotonically increasing. Therefore $f(x_0-\epsilon) < f(x_0) < f(x_0+\epsilon)$.
Now we can simply take $\delta = \min\{f(x_0)-f(x_0-\epsilon), f(x_0+s\epsilon) + f(x_0)\}$. For any $|y_1-y_0|<\delta$, let $f(x_1)=y_1$, then $x_0-\epsilon < x_1 < x_0 + \epsilon$ by monoticity. Thus $g$ is continuous.
\newpage


\section{Ross 30.2}
\subsection{a}
$\sin(0)-0=0$, so we attempt to use l'hospital's rule. Assume that the limit exists, then it must be equal to
$$\lim_{x \to 0}\frac{x^2}{\cos x} = \lim_{x \to 0}\frac{0}{1} = 0$$
Therefore the limit is 0

\subsection{b}
For this problem we need to use l'hoospital's rule 3 times
$$\lim_{x \to 0} \frac{\tan x-x}{x^3} = \lim_{x \to 0} \frac{\sec^2 x-1}{3x^2} = \lim_{x \to 0}\frac{2\tan(x)\sec^2(x)}{6x}$$
$$= \lim_{x \to 0}\frac{2 \sec^2x (\sec^2(x) + 2 \tan^2(x))}{6} = \frac{2}{6} = \frac{1}{3}$$
We used the chain rule for the second step, both the chain rule and the product rule for the third step.

\subsection{c}
We combine these fractions, then apply l'hospital's rule twice:
$$\lim_{x\to 0}\frac{x-\sin x}{x\sin x} = \lim_{x\to 0}\frac{1-\cos x}{\sin x + x\cos x} = \lim_{x\to 0}\frac{\sin x}{\cos x + \sin x - x \sin x} $$
$$= \lim_{x\to 0}\frac{0}{1} = 0$$

\subsection{d}

\end{document}
