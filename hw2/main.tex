\documentclass[12pt]{article}
\usepackage[usenames]{color} %used for font color
\usepackage{amssymb} %maths
\usepackage{amsmath} %maths
\usepackage[utf8]{inputenc} %useful to type directly diacritic characters
\usepackage{graphicx}
\usepackage [english]{babel}
\usepackage [autostyle, english = american]{csquotes}
\MakeOuterQuote{"}
\graphicspath{ {./} }
\newcommand{\Z}{\mathbb{Z}}
\newcommand{\N}{\mathbb{N}}
\newcommand{\R}{\mathbb{R}}
\newcommand{\Q}{\mathbb{Q}}
\newcommand{\prob}{\mathbb{P}}

\author{Tianshuang (Ethan) Qiu}
\begin{document}
\title{Homework 2}
\maketitle
\newpage

\section{Ross 4.15}
For this we first prove that $\frac{1}{n}>0 \forall n \in \N$. First we multiply both sides by $n$, since $n \in \N, n > 0$, the sign does not change.
\newline
We get $LHS = 1$ $RHS = 0$, since $1 > 0$ is an axiom, therefore we know that $1/n > 0$.
\newline
Now since we have proven $\frac{1}{n} \geq 0 \forall n \in \N$, by the ordered field axioms $a \leq b$.
\newline
Q.E.D.
\newpage


\section{Ross 4.16}
Let the set in question be denoted by $S$. We will prove the claim via contradiction. We break its negative down into two cases: $\sup S < a$ or $\sup S > a$.
\newline
For the former, let $x = \sup S | x \in \R$. By the denseness of rationals we see that there exists $q \in \Q s.t. x \leq q \leq a$. By the definition of this set we have $q \in S$. Therefore we have just found a member of this set that is greater than the supremum. This is a contradiction so $x$ cannot be less than a.
\newline
For the latter, we once again let $x = \sup S | x \in \R$. Consider $a$. $a<x$ and by definition of $S$, $\forall s \in S, s<a$. We have found an upperbound that is less than our supremum. That is a contradiction so $x$ cannot be greater than a.
\newline
$\sup S = a$ Q.E.D.
\newpage

\section{Ross 8.2}
\subsection{a}
Claim: $a_n \to 0$
\newline
Proof: $a_n = \frac{n}{n^2+1} \leq \frac{n+1/n}{n^2+1} = \frac{1}{n}$ ($n \neq 0$)
\newline
Let $\epsilon > 0$, we select our $N = \frac{1}{\epsilon}$. $\forall n>N, |a_n-0| < \frac{1}{n} < \epsilon$, thus the sequence converges.

\subsection{c}
Claim: $c_n \to \frac{4}{7}$
\newline
Proof: $|c_n- \frac{4}{7}| = |\frac{28n+21}{49n-35} - \frac{4(7n-5)}{49n-35}|$
$$= \frac{28n+21-28n+20}{49n-35}$$
$$= \frac{41}{49n-35} \leq \frac{41}{49n}$$
\newline
Let  $\epsilon > 0$, we select our $N = \frac{41}{49\epsilon}$. For all $n>N, |c_n- \frac{4}{7}| \leq \frac{41}{49n} < \epsilon$

\subsection{e}
Claim: $s_n \to 0$
\newline
Proof: $|s_n - 0| = |\frac{1}{n}sin n| \leq \frac{1}{n}$.
\newline
Let  $\epsilon > 0$, we select our $N = \frac{1}{\epsilon}$. For all $n>N, |s_n-0| \leq \frac{1}{n} < \epsilon$
\newpage


\section{Ross 8.5}
\subsection{a}
Let $\epsilon > 0$, since $a_n, b_n \to s$, $|a_n-s|<\epsilon \forall n > N_1$ and $|b_n-s|<\epsilon \forall n > N_2$.
\newline
Let $k> \max \{ N_1, N_2\}$, $a_k \leq s_k \leq b_k$. We subtract s from the expression and we have: $a_k-s \leq s_k -s \leq b_k-s$. Furthermore $|a_k-s|<\epsilon, |b_k - s| < \epsilon$.
\newline
Since $s_k-s$ is "sandwiched" between two expressions whose absolute values are less than epsilon, then $|s_k-s|<\epsilon$.
\newline
$s_n \to 0$, Q.E.D.

\subsection{b}
Claim: $lim s_n = 0$.
\newline
Proof:
Since the absolute value s strictly non-negative, $t_n \geq 0$. Let $\epsilon>0$, since $\lim t_n = 0$, $\exists N s.t. \forall k> N, t_k < \epsilon$. Then consider the seqence $d_n = 0$. Obviously $lim d_n = 0$.
\newline
$$0 = |d_k-0| \leq |s_k| = |s_k - 0| \leq t_k = |t_k - 0|$$
Therefore $s_n$ converges to 0 by squeeze lemma.
\newpage


\section{Ross 8.7}
\subsection{a}
Assume that this sequence $a_n$ converges, let $a_n \to k$. By our assumption $\exists N \in \R s.t. |a_n - k|<\epsilon \forall n > N$.
Since this is a cosine function it is cyclical, we can see that it goes 1, 0.5, -0.5, -1, -0.5, 0.5, ..., repeating ad infinitum.
\newline
Let $\epsilon = 0.1$. Select $t > N, t \bmod 6 \equiv 0$. By the pattern we observed above, we know that $a_t = 0$. Furthermore, we know that $a_{t+1} = 0.5$. By the definition of convergence we have $|a_t-k|<\epsilon$, $|a_{t+1}-k|< \epsilon$, subsituting the values we have calculated we have $|0-k|<0.1, |0.5-k|<0.1, |0-k|+|0.5-k|\leq 0.2$. However by the triangle property we know that $|0-k|+|0.5-k| \leq 0.5$. This is a contradiction, therefore our assuption is not correct.
\newline
$a_n$ does not converge. Q.E.D.


\subsection{b}
For this problem we simply need to show that the sequence is not bounded.
\newline
Assume that the sequence is bounded, and that there is a supremum $k$. By the Archimedean Principle $\exists n \in N s.t. n>k$. Consider $s_n$ (if $n$ is odd consider $s_{n+1}$), this term is greater than $k$. Therefore we have found a member in the set that is greater than the supremum. $\rightarrow \leftarrow$
\newline
The sequence is not bounded, therefore $s_n$ cannot converge. Q.E.D.


\subsection{c}
The sequence here is very similar to that in section (a). The pattern is 0, 0.5, 1, 0.5, 0, -0.5, -1, -0.5, ... . We can let $\epsilon = 0.1$ again and assume that it converges. So let $N \in \R s.t. |c_n - \lim c_n| < \epsilon \forall n > N$. Pick $i > N s.t. i \bmod 6 \equiv 0$. From the pattern that we observed, $c_{n+1} = 0.5$. By the triangle inequality we see that $\lim c_n$ cannot exist since we need the "two sides" (0.2) to be less than the other side (0.5).
\newline
We have found a contradiction, $c_n$ does not converge.
\newpage


\section{Ross 8.10}
Since $\lim s_n > a$, $\lim s_n - a > 0$. Let this value be $d$.
\newline
Consider $\epsilon = d$. Since the sequence converges we have $\exists N \in R s.t. \forall n > N, |s_n - \lim s_n|<\epsilon$. Since $\epsilon = \lim s_n - a$, we have
$$|s_n - \lim s_n| < \lim s_n - a$$
\newline
If $s_n \geq \lim s_n$, $s_n > a$ because $\lim s_n > a$.
\newline
Otherwise, $s_n < \lim s_n$. We can simplfy $|s_n - \lim s_n| < \lim s_n - a$ into $ \lim s_n - s_n < \lim s_n - a$, and by algebraic manipulation we have $s_n > a$.
\newline
In both cases $s_n > a$. Q.E.D.
\newpage


\section{Q7}
Claim: $\lim s_n = 1$
\newline
Let $\epsilon > 0$. Consider $a_n = 1$, $b_n = 1-\frac{1}{n}$. Obviously $a_n$ converges to 1.
\newline
For $b_n$, let $N = \frac{1}{\epsilon}$. $\forall k > N$, we have $|b_k - 1| = |1-\frac{1}{k}-1| = |-\frac{1}{k}| = \frac{1}{k} < \epsilon$. Therefore $b_n \to 1$
\newline
Since $\frac{1}{n} > 0 \forall n \in \N$, $(1-\frac{1}{n})<1$, so $\sqrt{(1-\frac{1}{n})} > (1-\frac{1}{n})$.
\newline
We have shown that $a_n \to 1$, $b_n \to 1$, and $b_n \leq s_n \leq a_n$. Therefore $s_n \to 1$ by squeeze theorem.
\newline
Q.E.D.
\end {document}
