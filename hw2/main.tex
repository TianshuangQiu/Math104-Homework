\documentclass[12pt]{article}
\usepackage[usenames]{color} %used for font color
\usepackage{amssymb} %maths
\usepackage{amsmath} %maths
\usepackage[utf8]{inputenc} %useful to type directly diacritic characters
\usepackage{graphicx}
\usepackage [english]{babel}
\usepackage [autostyle, english = american]{csquotes}
\MakeOuterQuote{"}
\graphicspath{ {./} }
\newcommand{\Z}{\mathbb{Z}}
\newcommand{\N}{\mathbb{N}}
\newcommand{\R}{\mathbb{R}}
\newcommand{\Q}{\mathbb{Q}}
\newcommand{\prob}{\mathbb{P}}

\author{Tianshuang (Ethan) Qiu}
\begin{document}
\title{Homework 2}
\maketitle
\newpage

\section{Ross 4.15}
For this we first prove that $\frac{1}{n}>0 \forall n \in \N$. First we multiply both sides by $n$, since $n \in \N, n > 0$, the sign does not change.
\newline
We get $LHS = 1$ $RHS = 0$, since $1 > 0$ is an axiom, therefore we know that $1/n > 0$.
\newline
Now since we have proven $\frac{1}{n} \geq 0 \forall n \in \N$, by the ordered field axioms $a \leq b$.
\newline
Q.E.D.
\newpage


\section{Ross 4.16}
Let the set in question be denoted by $S$. We will prove the claim via contradiction. We break its negative down into two cases: $\sup S < a$ or $\sup S > a$.
\newline
For the former, let $x = \sup S | x \in \R$. By the denseness of rationals we see that there exists $q \in \Q s.t. x \leq q \leq a$. By the definition of this set we have $q \in S$. Therefore we have just found a member of this set that is greater than the supremum. This is a contradiciton so $x$ cannot be less than a.
\newline
For the latter, we once again let $x = \sup S | x \in \R$. Consider $a$. $a<x$ and by definition of $S$, $\forall s \in S, s<a$. We have found an upperbound that is less than our supremum. That is a contradiction so $x$ cannot be greater than a.
\newline
$\sup S = a$ Q.E.D.
\newpage

\section{Ross 8.2}
\subsection{a}
Claim: $a_n \to 0$
\newline
Proof: $a_n = \frac{n}{n^2+1} \leq \frac{n+1/n}{n^2+1} = \frac{1}{n}$ ($n \neq 0$)
\newline
Let $\epsilon > 0$, we select our $N = \frac{1}{\epsilon}$. $\forall n>N, |a_n-0| < \frac{1}{n} < \epsilon$, thus the sequence converges.

\subsection{c}
Claim: $c_n \to \frac{4}{7}$
\newline
Proof: $|c_n- \frac{4}{7}| = |\frac{28n+21}{49n-35} - \frac{4(7n-5)}{49n-35}|$
$$= \frac{28n+21-28n+20}{49n-35}$$
$$= \frac{41}{49n-35} \leq \frac{41}{49n}$$
\newline
Let  $\epsilon > 0$, we select our $N = \frac{41}{49\epsilon}$. For all $n>N, |c_n- \frac{4}{7}| \leq \frac{41}{49n} < \epsilon$

\subsection{e}
Claim: $s_n \to 0$
\newline
Proof: $|s_n - 0| = |\frac{1}{n}sin n| \leq \frac{1}{n}$.
\newline
Let  $\epsilon > 0$, we select our $N = \frac{1}{epsilon}$. For all $n>N, |s_n-0| \leq \frac{1}{n} < \epsilon$
\newpage


\section{Ross 8.5}
\subsection{a}





\end {document}
