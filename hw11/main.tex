\documentclass[12pt]{article}
\usepackage[usenames]{color} %used for font color
\usepackage{amsmath, amssymb, amsthm}
\usepackage{wasysym}
\usepackage[utf8]{inputenc} %useful to type directly diacritic characters
\usepackage{graphicx}
\usepackage{caption}
\usepackage{subcaption}
\usepackage{float}
\usepackage{mathtools}
\usepackage [english]{babel}
\usepackage [autostyle, english = american]{csquotes}
\MakeOuterQuote{"}
\graphicspath{ {./} }
\newcommand{\Z}{\mathbb{Z}}
\newcommand{\N}{\mathbb{N}}
\newcommand{\R}{\mathbb{R}}
\newcommand{\Q}{\mathbb{Q}}
\newcommand{\prob}{\mathbb{P}}
\newcommand{\degrees}{^{\circ}}
\DeclarePairedDelimiter\ceil{\lceil}{\rceil}
\DeclarePairedDelimiter\floor{\lfloor}{\rfloor}

\author{Tianshuang (Ethan) Qiu}
\begin{document}
\title{Math 104, HW11}
\maketitle
\newpage

\section{Q1}
Assume that $\lim_{n \to \infty} \sup \{|f_n(x)-f(x)|\} = 0$, then let $\epsilon > 0$. By our assumption there exists $N \in \R$ such that $\sup \{ |f_n(x) - f(x)| \}<\epsilon$ for all $x \in S, n > N$. Since the supremum is less than $\epsilon$, every member in that set must also be less than epsilon, which is the definition for uniform convergence. Therefore $f_n \to f$ uniformly.
\newline
Now assume that $f_n \to f$ uniformly. Assume that $\lim_{n \to \infty} \sup \{|f_n(x)-f(x)|\} = k > 0$, then let $\epsilon = k/2$. Since $f_n \to f$ uniformly for any $\epsilon > 0$ we can find $N \in \R$ such that $|f_m(x)-f(x)|<\epsilon$ for all $x \in S, m > N$.
Now consider this $m$. It implies that the supremum of the set $\{|f_m(x)-f(x)|\}$ is less than or equal to $k/2$, which is less than the supremum of the set as $n$ approaches infinity.
Since we know that $\lim \sup y_n \leq \sup y_n$, we have reached a contradiction, our assumption is incorrect and $\lim_{n \to \infty} \sup \{|f_n(x)-f(x)|\} = 0$. Thus we have proven the converse.
\newpage


\section{Q2}
We claim that $f_n \to f: [0,1] \to \R$, $f(x)=x^2$ uniformly.
\newline
Let $\epsilon > 0$, choose $N = \max \{1, \frac{4}{\epsilon}\}$, consider $n > N$. Take $x \in [0,1]$, $|f_n(x)-f(x)| = |(x-\frac{1}{n})^2-x^2|$
$$= |x^2+\frac{1}{n^2}-\frac{2x}{n}-x^2| = |\frac{1}{n^2}-\frac{2x}{n}|$$
Since we know that $n\geq 1$ and $x \in [0,1]$, we know that $\frac{1}{n^2} < \frac{2x}{n}$, so their difference is (nonstrictly) less than $\frac{2x}{n}$
\newline
Now since $n > N$, $\frac{2x}{n} \leq \frac{2}{n} < \frac{\epsilon}{2} < \epsilon$.
Therefore $|f_n(x)-f(x)|<\epsilon$, and we have proven that $f_n \to f$ uniformly.
\newpage


\section{Q3}
\subsection{Pointwise Convergent}
Let $\epsilon > 0$ and $x \in [0,1]$, choose $N = \frac{\epsilon}{2(1-x)}$. For all $n > N$, consider $|f_n(x)-f(x)|= |nx^n(1-x)-0| \leq |Nx^n(1-x)| \leq |\frac{\epsilon}{2}x^n| \leq |\epsilon/2| < \epsilon$
\newline
Thus given any point in $[0,1]$ and $\epsilon$, we can find a $N$ such that $|f_n(x)-f(x)|<\epsilon$

\subsection{Not Unif. Convergent}
Choose $\epsilon = \frac{1}{10e}$, let $N \in \R$. Consider $m > N$. Since $f_m(x)$ is a polynomial we know that it is differentiable. Furthermore, since $f_m(0)=0$, $f_m(1)=0$, the maximum of $f_m(x)$ in $[0,1]$ must appear in $(0,1)$, then its derivative must be 0. $f_m(x)=mx^m(1-x)=mx^m-mx^{m+1}$.
$$f_m'(x)= m^2x^{m-1}-m(m+1)x^{m} = 0$$
$$m^2x^{m-1}=m^2x^m+mx^m$$
$$m^2 = m^2x+mx$$
$$x = \frac{m}{m+1}$$
We find that $f_m'(x)=0$ when $x = \frac{m}{m+1}$, now we attempt to find $f_m(x)$ at this $x$. $f_m(x)=(\frac{m}{m+1})^m \frac{m}{m}$
\newline
Since we are intrested in the behavior of this value as $N \to \infty$, we know that the limit of the maximum of this function is $\lim_{m \to \infty}(\frac{m}{m+1})^m = \frac{1}{e}$. Recall that our $\epsilon = \frac{1}{10e}$ which is less than the maximum of the function $\lim_{m \to \infty} f_m$, therefore there is no $N$ that can satisfy the requirements for uniform convergence.
Thus it is not uniformly convergent.
$\blacksquare$
\newpage


\section{Ross 25.7}
Since $|\cos(k)| \leq 1$, $|\frac{1}{n^2}\cos(nx)| \leq \frac{1}{n^2}$, and since $\sum \frac{1}{n^2}$ converges by the $p$ series test, $\sum \frac{1}{n^2}\cos(nx)$ converges uniformly as well by the M-test.
\newline
Furthermore, since $\frac{1}{n^2}\cos(nx)$ is continuous for all $n \in \N$, and it converges uniformly, it must converge to a continuous function.
\newline
$\blacksquare$
\newpage


\section{Q5}
\subsection{Pointwise Convergent}
Let $x \in (0,1), \epsilon > 0$, pick $N = \log_x((1-x)\epsilon) = \frac{\ln((1-x)\epsilon)}{\ln(x)}$, consider an arbitrary $n > N$. $f_n(x) = \sum_{k=0}^n g_k =  \sum_{k=0}^n x_k$
$$|f_n(x)-f_x| = |(\sum_{k=0}^n x_k) - \frac{1}{1-x}|$$
We apply the formula for a geometric sum to the former, since $x<1$, $(1-x^n)$ is positive
$$= |\frac{1-x^n}{1-x} - \frac{1}{1-x}| = |\frac{-x^n}{1-x}| = \frac{x^n}{1-x}$$
Now recall our $N$, we know that the difference is strictly less than $\frac{x^N}{1-x} = \frac{\epsilon (1-x)}{1-x} = \epsilon$
\newline
Thus we have proven that $\sum_k=0^\infty g_k \to f$ pointwise.

\subsection{Not Uniformly Convergent}
We will use the fact that finite sums of finite numbers cannot be infinite to create a contradiction.
\newline
First for our function $f$, as $x \to 1$, $(1-x) \to 0$, and thus $\frac{1}{1-x} \to infty$. Assume that $f_n \to f$ uniformly. Then choose $\epsilon = 1$, by our assumption there exists $N \in \R$ such that for all $m > n$, $|f_m(x)-f(x)|<\epsilon$ for all $x \in (0,1)$.
\newline
Since $f_m$ is the sum of $m+1$ terms starting from $x^0$, and $0 < x < 1$, we know that $\lim_{x \to 1} f_m \leq (m+1) < \infty$. Now we just need to find this point.
\newline
We are looking for a value such that the difference is equal to our epislon: $f(x)=m+1+1 = m+2$. Now we try to find this point in $(0,1)$:
$$\frac{1}{1-x}=m+2$$
$$1 = (1-x)(m+2) = m +2 -xm -2x$$
$$x(m+2) = m+1$$
$$x = \frac{m+1}{m+2}$$
Since $m>0$, $x \in (0,1)$, and at this point, $|f_m(x)-f(x)|=\epsilon$, and we have a contradiction. Therefore our assumption is incorrect and $\sum_k=0^\infty g_k \not \to f$ uniformly.
\newpage


\section{Ross 25.3}
\subsection{a}
We claim that the series of functions converge go $f: \R \to \R$, $f(x) = \frac{1}{2}$. Consider $\lim_{n \to \infty} \sup \{ |f_n(x)-f(x)|\} = \lim_{n \to \infty} \sup \{ |\frac{n + \cos x}{2n + \sin^2x}-\frac{1}{2}|\}$
\newline
Since both cosine and sine are non-strictly between 0 and 1, the greatest $\frac{n + \cos x}{2n + \sin^2x}$ can be is $\frac{n+1}{2n}$, and the smallest it can be is $\frac{n}{2n+1}$, therefore we have the inequality:
$$|\frac{n + \cos x}{2n + \sin^2x}-\frac{1}{2}| \leq \max \{\frac{n+1}{2n}, \frac{n}{2n+1}\} - \frac{1}{2}$$
Now consider $\lim_{n \to \infty}\frac{n+1}{2n} = \frac{1}{2}$ since the difference $|\frac{n+1}{2n}-\frac{1}{2}| = |\frac{1}{2n}|$ approaches 0 as $n \to \infty$. Similarly $\lim_{n \to \infty}\frac{n}{2n+1} = \frac{1}{2}$
\newline
Therefore we know that $\max \{\frac{n+1}{2n}, \frac{n}{2n+1}\} - \frac{1}{2} = 0$, thus
$$\lim_{n \to \infty} \sup \{ |f_n(x)-f(x)|\} = 0$$
From our theorem in Q1 we know that $f_n \to f$ uniformly.

\subsection{b}
By the theorem that if $f_n$ are continuous and $f_n \to f$ uniformly, then $\lim_{n \to \infty} \int_a^b f_n(x) dx= \int_a^b f(x)dx$, we know that we can subsitute the integral for
$$\int_2^7 \frac{1}{2}dx = \frac{7}{2}-\frac{2}{2} = 2.5$$
\newpage


\section{Ross 23.1}
\subsection{a}
We compute $\lim \sup (a_n)^\frac{1}{n} = \lim \sup (n^2)^\frac{1}{n} = \lim \sup (n^\frac{1}{n})^2 = 1$
Therefore $R = 1/1 = 1$
\newline
Now we check -1: $\sum (-1)^nn^2$ does not converge by the alternating series test.
\newline
For $x=1$, $\lim_{n \to \infty} n^2 \not 0$, therefore it diverges.
\newline
Finally, the intereval of convergence is found to be $(-1,1)$

\subsection{c}
We compute $\lim \sup (a_n)^\frac{1}{n} = \lim \sup (\frac{2^n}{n^2})^\frac{1}{n} = \lim \sup \frac{2^{n\frac{1}{n}}}{1} = 2$
Therefore $R = 1/2 = \frac{1}{2}$
\newline
Now we check $-\frac{1}{2}$: $\sum (-1)^n\frac{1}{n^2}$ converges by the alternating series test.
\newline
For $x= \frac{1}{2}$, $\sum \frac{1}{n^2}$ converges by p-series test.
\newline
Finally, the intereval of convergence is found to be $[-\frac{1}{2},\frac{1}{2}]$

\subsection{e}
We compute $\lim \sup (a_n)^\frac{1}{n} = \lim \sup (\frac{2^n}{n!})^\frac{1}{n} = \lim \sup \frac{2}{n!^{\frac{1}{n}}} = 0$
Therefore $R = \infty$
\newline
The interval of convergence is defined to be $(-\infty, \infty)$

\subsection{g}
We compute $\lim \sup (a_n)^\frac{1}{n} = \lim \sup (\frac{3^n}{n4^n})^\frac{1}{n} = \lim \sup (\frac{3}{4})^{n\frac{1}{n}}\frac{1}{1} = \frac{3}{4}$
Therefore $R = 1/(3/4) = \frac{4}{3}$
\newline
Now we check $-\frac{4}{3}$: $\sum (-1)^n\frac{1}{n}$ converges by the alternating series test.
\newline
For $x= \frac{4}{3}$, $\sum \frac{1}{n}$ does not converge by p-series test.
\newline
Finally, the intereval of convergence is found to be $[-\frac{4}{3},\frac{4}{3})$


\end{document}
