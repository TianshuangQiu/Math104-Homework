\documentclass[12pt]{article}
\usepackage[usenames]{color} %used for font color
\usepackage{amsmath, amssymb, amsthm}
\usepackage{wasysym}
\usepackage[utf8]{inputenc} %useful to type directly diacritic characters
\usepackage{graphicx}
\usepackage{caption}
\usepackage{subcaption}
\usepackage{float}
\usepackage{mathtools}
\usepackage [english]{babel}
\usepackage [autostyle, english = american]{csquotes}
\MakeOuterQuote{"}
\graphicspath{ {./} }
\newcommand{\Z}{\mathbb{Z}}
\newcommand{\N}{\mathbb{N}}
\newcommand{\R}{\mathbb{R}}
\newcommand{\Q}{\mathbb{Q}}
\newcommand{\prob}{\mathbb{P}}
\newcommand{\degrees}{^{\circ}}
\DeclarePairedDelimiter\ceil{\lceil}{\rceil}
\DeclarePairedDelimiter\floor{\lfloor}{\rfloor}

\author{Tianshuang (Ethan) Qiu}
\begin{document}
\title{Math 104, HW12}
\maketitle
\newpage

\section{Q1}
\subsection{a}
First we know that $\sqrt{x^2} = |x|$. Since we know that $\frac{1}{n} \to 0$ and $\frac{1}{n^2} \to 0$, we can show uniform convergence by the following.
\newline
Let $\epsilon > 0$, pick $N$ such that for all $n > N$, $|\frac{1}{n}| < \epsilon^2$. Now since $|x| < 1$ and the domain of the square root being positive,
$$f_n(x) = \sqrt{x^2 + \frac{1}{n}} \leq \sqrt{x^2 + \frac{2}{\sqrt{n}}|x| + \frac{1}{n}} \leq \sqrt{(x+\frac{1}{\sqrt{n}})^2}$$
By our first statement the above expression is equal to $|(x+\frac{1}{\sqrt{n}})|$. By our definition of $N$,
$$||(x+\frac{1}{\sqrt{n}})| - |x|| \leq |x+\frac{1}{\sqrt{n}}-x| = |\frac{1}{\sqrt{n}}| < \epsilon$$
Thus we have $f_n \to |x|$ uniformly.

\subsection{b}
$f_n(x) = \sqrt{x^2 + \frac{1}{n}}$, and by the power rule we know that
$$f_n'(x) = \frac{x}{\sqrt{x^2+\frac{1}{n}}}$$

\subsection{c}
Define $g: (-1,1) \to \R$, $g(x) = -1 $ for $x \in (-1,0)$, $g(x) = 1 $ for $x \in (0, 1)$, $g(0)=0.$

\end{document}
