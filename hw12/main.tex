\documentclass[12pt]{article}
\usepackage[usenames]{color} %used for font color
\usepackage{amsmath, amssymb, amsthm}
\usepackage{wasysym}
\usepackage[utf8]{inputenc} %useful to type directly diacritic characters
\usepackage{graphicx}
\usepackage{caption}
\usepackage{subcaption}
\usepackage{float}
\usepackage{mathtools}
\usepackage [english]{babel}
\usepackage [autostyle, english = american]{csquotes}
\MakeOuterQuote{"}
\graphicspath{ {./} }
\newcommand{\Z}{\mathbb{Z}}
\newcommand{\N}{\mathbb{N}}
\newcommand{\R}{\mathbb{R}}
\newcommand{\Q}{\mathbb{Q}}
\newcommand{\prob}{\mathbb{P}}
\newcommand{\degrees}{^{\circ}}
\DeclarePairedDelimiter\ceil{\lceil}{\rceil}
\DeclarePairedDelimiter\floor{\lfloor}{\rfloor}

\author{Tianshuang (Ethan) Qiu}
\begin{document}
\title{Math 104, HW12}
\maketitle
\newpage

\section{Q1}
\subsection{a}
First we know that $\sqrt{x^2} = |x|$. Since we know that $\frac{1}{n} \to 0$ and $\frac{1}{n^2} \to 0$, we can show uniform convergence by the following.
\newline
Let $\epsilon > 0$, pick $N$ such that for all $n > N$, $|\frac{1}{n}| < \epsilon^2$. Now since $|x| < 1$ and the domain of the square root being positive,
$$f_n(x) = \sqrt{x^2 + \frac{1}{n}} \leq \sqrt{x^2 + \frac{2}{\sqrt{n}}|x| + \frac{1}{n}} \leq \sqrt{(x+\frac{1}{\sqrt{n}})^2}$$
By our first statement the above expression is equal to $|(x+\frac{1}{\sqrt{n}})|$. By our definition of $N$,
$$||(x+\frac{1}{\sqrt{n}})| - |x|| \leq |x+\frac{1}{\sqrt{n}}-x| = |\frac{1}{\sqrt{n}}| < \epsilon$$
Thus we have $f_n \to |x|$ uniformly.

\subsection{b}
$f_n(x) = \sqrt{x^2 + \frac{1}{n}}$, and by the power rule we know that
$$f_n'(x) = \frac{x}{\sqrt{x^2+\frac{1}{n}}}$$

\subsection{c}
Define $g: (-1,1) \to \R$, $g(x) = -1 $ for $x \in (-1,0)$, $g(x) = 1 $ for $x \in (0, 1)$, $g(0)=0$
\newline
For any $x < 0$, let $a_n = \frac{1}{f_n'(x)} = \frac{\sqrt{x^2+\frac{1}{n}}}{x}$. By part $a$ we know that $\sqrt{x^2+\frac{1}{n}} \to |x|$ uniformly.
Let $\epsilon>0$ we pick $N$ such that $|\sqrt{x^2+\frac{1}{n}}-|x||<\epsilon x - x$, therefore
$$|a_n-(-1)| = \frac{\sqrt{x^2+\frac{1}{n}}+x}{x} = \frac{|\sqrt{x^2+\frac{1}{n}}-|x||}{x} < \frac{\epsilon x}{x} = \epsilon$$
If $x>0$, the same is true because the denominator is now positive and the absolute value sign should be flipped. Finally, if $x=0$, $f_n'(x)=g(x)=0$. Thus $f_n'(x) \to g(x)$ pointwise.
\newline
Since $f_n'(x)$ is a polynomial divided by a non-zero polynomial, it is continuous for all values of $n$ and for all values of $x \in (-1,1)$. However $g(x)$ is not continuous at $0$. By our theorem about uniform convergence of continuous functions, we know that $f_n'(x) \not \to g(x)$ uniformly.
\newpage


\section{Q2}
For any $x \in (-1,1)$, shrink the domain of our power series to $[(x+1)/2, (1-x)/2]$. Now since both of these end points are within our radius of convergence, $\sum_{n=0}^{\infty}x^n \to \frac{1}{1-x}$ uniformly.
\newline
Then, we can take the derivative of both sides:
$$(\sum_{n=0}^{\infty}x^n \to \frac{1}{1-x}) = \sum_{n=1}^{\infty}nx^n$$
$$(\frac{1}{1-x})' = \frac{x}{(1-x)^2}$$
Thus $\sum_{n=1}^{\infty}nx^n \to \frac{x}{(1-x)^2}$
\newpage


\section{Q3}

\end{document}
