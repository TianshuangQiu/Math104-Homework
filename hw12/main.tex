\documentclass[12pt]{article}
\usepackage[usenames]{color} %used for font color
\usepackage{amsmath, amssymb, amsthm}
\usepackage{wasysym}
\usepackage[utf8]{inputenc} %useful to type directly diacritic characters
\usepackage{graphicx}
\usepackage{caption}
\usepackage{subcaption}
\usepackage{float}
\usepackage{mathtools}
\usepackage [english]{babel}
\usepackage [autostyle, english = american]{csquotes}
\MakeOuterQuote{"}
\graphicspath{ {./} }
\newcommand{\Z}{\mathbb{Z}}
\newcommand{\N}{\mathbb{N}}
\newcommand{\R}{\mathbb{R}}
\newcommand{\Q}{\mathbb{Q}}
\newcommand{\prob}{\mathbb{P}}
\newcommand{\degrees}{^{\circ}}
\DeclarePairedDelimiter\ceil{\lceil}{\rceil}
\DeclarePairedDelimiter\floor{\lfloor}{\rfloor}

\author{Tianshuang (Ethan) Qiu}
\begin{document}
\title{Math 104, HW12}
\maketitle
\newpage

\section{Q1}
\subsection{a}
First we know that $\sqrt{x^2} = |x|$. Since we know that $\frac{1}{n} \to 0$ and $\frac{1}{n^2} \to 0$, we can show uniform convergence by the following.
\newline
Let $\epsilon > 0$, pick $N$ such that for all $n > N$, $|\frac{1}{n}| < \epsilon^2$. Now since $|x| < 1$ and the domain of the square root being positive,
$$f_n(x) = \sqrt{x^2 + \frac{1}{n}} \leq \sqrt{x^2 + \frac{2}{\sqrt{n}}|x| + \frac{1}{n}} \leq \sqrt{(x+\frac{1}{\sqrt{n}})^2}$$
By our first statement the above expression is equal to $|(x+\frac{1}{\sqrt{n}})|$. By our definition of $N$,
$$||(x+\frac{1}{\sqrt{n}})| - |x|| \leq |x+\frac{1}{\sqrt{n}}-x| = |\frac{1}{\sqrt{n}}| < \epsilon$$
Thus we have $f_n \to |x|$ uniformly.

\subsection{b}
$f_n(x) = \sqrt{x^2 + \frac{1}{n}}$, and by the power rule we know that
$$f_n'(x) = \frac{x}{\sqrt{x^2+\frac{1}{n}}}$$

\subsection{c}
Define $g: (-1,1) \to \R$, $g(x) = -1 $ for $x \in (-1,0)$, $g(x) = 1 $ for $x \in (0, 1)$, $g(0)=0$
\newline
For any $x < 0$, let $a_n = \frac{1}{f_n'(x)} = \frac{\sqrt{x^2+\frac{1}{n}}}{x}$. By part $a$ we know that $\sqrt{x^2+\frac{1}{n}} \to |x|$ uniformly.
Let $\epsilon>0$ we pick $N$ such that $|\sqrt{x^2+\frac{1}{n}}-|x||<\epsilon x - x$, therefore
$$|a_n-(-1)| = \frac{\sqrt{x^2+\frac{1}{n}}+x}{x} = \frac{|\sqrt{x^2+\frac{1}{n}}-|x||}{x} < \frac{\epsilon x}{x} = \epsilon$$
If $x>0$, the same is true because the denominator is now positive and the absolute value sign should be flipped. Finally, if $x=0$, $f_n'(x)=g(x)=0$. Thus $f_n'(x) \to g(x)$ pointwise.
\newline
Since $f_n'(x)$ is a polynomial divided by a non-zero polynomial, it is continuous for all values of $n$ and for all values of $x \in (-1,1)$. However $g(x)$ is not continuous at $0$. By our theorem about uniform convergence of continuous functions, we know that $f_n'(x) \not \to g(x)$ uniformly.
\newpage


\section{Q2}
For any $x \in (-1,1)$, shrink the domain of our power series to $[(x+1)/2, (1-x)/2]$. Now since both of these end points are within our radius of convergence, $\sum_{n=0}^{\infty}x^n \to \frac{1}{1-x}$ uniformly.
\newline
Then, we can take the derivative of both sides:
$$(\sum_{n=0}^{\infty}x^n \to \frac{1}{1-x}) = \sum_{n=1}^{\infty}nx^n$$
$$(\frac{1}{1-x})' = \frac{x}{(1-x)^2}$$
Thus $\sum_{n=1}^{\infty}nx^n \to \frac{x}{(1-x)^2}$
\newpage


\section{Q3}
Let $x \in \R$, since $e^x = \sum_{k=0}^{\infty}\frac{x^k}{k!}$,
$$(e^x)' = \sum_{j=1}^{\infty}\frac{jx^{j-1}}{j!}$$
However, since $\frac{jx^{j-1}}{j!} = \frac{x^{j-1}}{(j-1)!}$, since $j$ begins at $1$ and $k$ at $0$, each term can be matched bijectively to the first sum.
\newline
Thus they are the same sum. $\blacksquare$
\newpage


\section{Q4}
Since $f(x)=e^{-x^2}$, by the last problem we know that it is equal to the sum of $\sum_{k=0}^{\infty}\frac{(-x^2)^k}{k!} = \sum_{k=0}^{\infty}(-1)^k\frac{x^{2k}}{k!}$
\newline
Now we integrate our power series term by term.
$$\int_0^y \sum_{k=0}^{\infty}(-1)^k\frac{x^{2k}}{k!} = \sum_{k=0}^{\infty}(-1)^{k}\frac{1}{(1+2k)k!}y^{1+2k}$$
Let $f_n$ be the above series, so $f_n' \to e^{{x^2}}$, therefore $f_n \to \int_0^x e^{{x^2}}$.
\newpage


\section{Q5}
Since $f'(0)$ does not exist, the Taylor Polynomial of degree $n \geq 1$ does not exist. Thus the Taylor Series is $0$, and we can let $\epsilon = 0.1$, at $0.5$, $|f(x)-0| = 0.4 > \epsilon$. Thus the Taylor Series does not converge to $f$, and therefore there is no power series that converge.
\newpage


\section{Q6}
Base case: $n=1$. We apply the chain rule:
$$(e^{\frac{1}{x^2}})' = e^{\frac{1}{x^2}}\frac{-2}{x^3}$$
Finally, let $a_{1,k} = 0$ for all $k \not = 3$, and $a_{1,3}=-2$, and our base case holds.
\newline
Inductive Case: let our formula hold for all $f^m(x) (m \leq n)$ for some $n \in \N$, then consider $f^{n+1}(x)$. We apply the product rule:
$$f^{n+1}(x) = (e^{\frac{1}{x^2}})'(\sum_{k=1}^{3n}\frac{a_{n,k}}{x^k}) + (e^{\frac{1}{x^2}})(\sum_{k=1}^{3n}\frac{a_{n,k}}{x^k})'$$
$$= e^{\frac{1}{x^2}}\frac{-2}{x^3}(\sum_{k=1}^{3n}\frac{a_{n,k}}{x^k}) + e^{\frac{1}{x^2}}(\sum_{k=1}^{3n}\frac{-ka_{n,k}}{x^{k+1}})
= e^{\frac{1}{x^2}}(\sum_{k=1}^{3n}\frac{-2a_{n,k}}{x^{k+3}}) + e^{\frac{1}{x^2}}(\sum_{k=1}^{3n}\frac{-ka_{n,k}}{x^{k+1}})$$
By factoring out $e^{\frac{1}{x^2}}$ we can see that the numerator contains constants, which is under the scope of $a_{n,k}$, the denominator has greatest possible degree of $x^{k+3}$. Since our sum can now go to $3(n+1) = 3n+3$, which is $3$ more than $n$, the denominator can also be covered by our formula.
\newline
Thus we have proven the inductive case, and the proof is complete.
\newpage


\section{Q7}
\subsection{a}
$\lim_{x \to 0} x^k = 0$, and $\lim_{x \to 0} \frac{1}{x^2} = \infty$ so $\lim_{x \to 0} e^{1/x^2} = 0$. Furthermore, both functions are differentiable at $0$, thus we may apply l'hospital's rule to this problem. Let $y = \frac{1}{x}$, so
$$\lim_{x \to 0} \frac{e^{-\frac{1}{x^2}}}{x^k} = \lim_{x \to 0} \frac{e^{\frac{1}{x^2}}\frac{-2}{x^3}}{kx^{k-1}} = \frac{-2e^{\frac{1}{x^2}}}{kx^{k}}$$

\end{document}
