\documentclass[12pt]{article}
\usepackage[usenames]{color} %used for font color
\usepackage{amsmath, amssymb, amsthm}
\usepackage{wasysym}
\usepackage[utf8]{inputenc} %useful to type directly diacritic characters
\usepackage{graphicx}
\usepackage{caption}
\usepackage{subcaption}
\usepackage{float}
\usepackage{mathtools}
\usepackage [english]{babel}
\usepackage [autostyle, english = american]{csquotes}
\MakeOuterQuote{"}
\graphicspath{ {./} }
\newcommand{\Z}{\mathbb{Z}}
\newcommand{\N}{\mathbb{N}}
\newcommand{\R}{\mathbb{R}}
\newcommand{\Q}{\mathbb{Q}}
\newcommand{\prob}{\mathbb{P}}
\newcommand{\degrees}{^{\circ}}
\DeclarePairedDelimiter\ceil{\lceil}{\rceil}
\DeclarePairedDelimiter\floor{\lfloor}{\rfloor}

\author{Tianshuang (Ethan) Qiu}
\begin{document}
\title{Math 104, HW8}
\maketitle
\newpage

\section{Q1}
For $x \not = 1$, $\frac{1}{x}$ is differentiable, and according to the inverse theorem, it is equal to $-\frac{1}{x^2}$
\newline
Since $\sin'(x) = \cos(x)$, and $\sin$ is well defined on all of $\R$ (the codomain of $\frac{1}{x}: \R \setminus 0 \to \R$), we can apply the chain rule to the second factor: $(\sin(\frac{1}{x}))' = \cos(\frac{1}{x})(-\frac{1}{x^2})$
\newline
Now since $x^2: \R \to \R$ is continuous in all its domain, we attempt to differentiate at an arbitrary point $x_0$:
$$\lim_{x \to x_0}\frac{(x^2)-(x_0^2)}{x-x_0} = \lim_{x \to x_0}\frac{(x+x_0)(x-x_0)}{x-x_0} = \lim_{x \to x_0}(x+x_0) = 2x_0$$
Therefore $(x^2)'=2x$
\newline
Finally we use the product rule and the derivative is $$2x\sin(\frac{1}{x}) + x^2\cos(\frac{1}{x})(-\frac{1}{x^2})= 2x\sin(\frac{1}{x})-\cos(\frac{1}{x})$$
\newpage


\section{Q2}
We claim that $f'(0)$ exists and is equal to $0$.
\newline
Consider the definition of the derivative: $$\lim_{x \to 0}\frac{x^2\sin(\frac{1}{x})-0}{x-0}$$
Since this function is defined on every point but 0, we have $f': \R \setminus 0 \to \R$
Now we can simplify to:
$$\lim_{x \to 0}x \sin(\frac{1}{x}) $$
Now we apply the squeeze theorem with $-1 \leq \sin(s) \leq 1$, and
$$-x \leq x \sin(\frac{1}{x}) \leq x$$
Since $-x, x$ both converge to 0, our derivative also converges to 0.
\newpage


\section{Q3}
We will use the fact that the derivative at $\R \setminus 0$ contains $\cos(\frac{1}{x})$, which fluctuates rapidly when $x$ is close to 0 to show a contradiction.
\newline
Assume that the derivative function $f'$ is continuous on $\R$. So let $\epsilon = 0.1, x_0=0$, then by our assumption there exists $\delta$ such that $|x-x_0|<\delta$ implies $|f'(x)-f'(0)|<\epsilon$
\newline
We have shown above that $f'(0)=0$, so we expand $|f'(x)-f'(0)|$:
$$=|2x\sin(\frac{1}{x})-\cos(\frac{1}{x})-0| \leq |2x\sin(\frac{1}{x})|+|\cos(\frac{1}{x})| \leq |\cos(\frac{1}{x})|$$
\newline
By the Archimedean Principle we know that there exists $n$ such that $\frac{1}{n}<\delta$. Set $x = \frac{1}{2\pi n} <\frac{1}{n}<\delta$. Consider $f'(x) = |\cos(2\pi n)| = 1 > \epsilon$.
\newline
Therefore when $\epsilon=0.1, x_0=0$, we have found an $x$ such that for any $\delta > 0$, though $|x-x_0|<\delta$, $|f'(x)-f'(x_0)|>\epsilon$
\newline
Thus the function is not continuous.
$\blacksquare$
\newpage


\section{Q4}

\subsection{a}
Base case:
\newline
$(x^1)'$ By the definition of the derivative we know that $f'(a)=\lim_{x \to a} \frac{x-a}{x-a} = 1$ since the numerator and the denominator cancel.
\newline
Inductive step:
\newline
Assume that $(x^n)' = nx^{n-1}$ for some $n \in \N$ Consider $x^{n+1}$
\newline
We write it as $x^nx$, now we apply can the product rule. We have shown in the base case that $x'=1$, so we have
$$(x^{n+1})' = nx^{n-1}x + 1x^{n} = (n+1)x^{n+1}$$
Thus we have shown the inductive step. $\blacksquare$


\subsection{b}
We rewrite $(\frac{f}{g})' = (f\frac{1}{g})'$ Now we attempt to apply the chain rule to $(\frac{1}{g})'$
\newline
Since $g$ has codomain of $\R$ and is differentiable at $a$, we can use the chain rule.
$$(\frac{1}{g})' = -\frac{1}{(g(a))^2}g'(a)$$
Now we multiply $f(a)$ in with the product rule
$$(\frac{f}{g})' = f'(a)\frac{1}{g(a)}-f(a)\frac{1}{(g(a))^2}g'(a)
= \frac{f'(a)g(a)}{(g(a))^2}-\frac{f(a)g'(a)}{(g(a))^2}$$
Thus it is proven.
\newpage


\section{Ross 29.5}
My key observation to this problem is that when $x,y$ are close to each other, $(x-y)^2$ becomes very small. So in order to be smaller than or equal to this, $|f(x)-f(y)|$ must also be able to get arbitrarily small.
\newline
Let $\epsilon > 0$, for any $a \in \R$, we can simply pick our delta to be $\min\{1, \sqrt(\epsilon/2)\}$ Let $|x-a|<\delta$, then $(x-a)^2 < \delta ^2 = \epsilon/2$. Now by the statement of the problem $|f(x)-f(a)|\leq (x-a)^2 <\epsilon/2$. Therefore $f$ is be continuous everywhere.
\newline
Now consider the absolute value of the derivative of this function:
$$|f'(a)| = \lim_{x \to a}\frac{|f(x)-f(a)|}{|x-a|}$$
We can slightly modify the specification of the problem. Since $x^2 = (-x)^2$, we can rewrite the function assumption to be $|f(x)-f(y)|\leq |x-y|^2$
\newline
Now we have $|f'(a)| \leq \lim_{x \to a}x - a = 0$. Since $|f'(a)| \geq 0$, we have $|f'(a)| = 0$. Thus we have shown that the function has a derivative of $0$ everywhere, and it is therefore constant.
$\blacksquare$
\newpage


\section{Ross 29.13}
Consider $h(x)=g(x)-f(x)$. $h(0) = g(0)-f(0)=0, h'(x)=g'(x)-f'(x)$. By the specification of the problem we know that $h'(x)\geq 0 $ when $x \geq 0$
\newline
Now consider $h(y)$ with $y\geq 0$. Assume that there is a point $z \geq 0$ such that $g(z)<f(z)$, so $h(z)= g(z)-f(z) < 0$. Since $h(0)=0$, and $h$ is differentiable, by the Mean Value Theorem we know that there exists a point $p \in (0,z)$ where $h'(z)=\frac{h(z)-0}{z-0}<0$. However by our assumption we know that $h'(z) \geq 0 \forall z \geq 0$. We have reached a contradiction.
\newline
Therefore our assumption is incorrect, $g(x)\geq f(x)$ for all $x \geq 0$.
\newpage


\section{Ross 29.17}

\subsection{$h$ is differentiable implies equal derivative and limit}
Let $h$ be differentiable at $a$. Then by the definition of the derivative we have $\lim_{x \to a} \frac{h(x)-h(a)}{x-a}$ exists, which implies that $\lim_{x \to a} h(x)$ exists.
Since the limit exists, for any $\epsilon > 0$, $\exists \delta > 0$ such that $|x-a|<\delta$ implies $|h(x)-h(a)|<\epsilon$. $h(a)=g(a)$ When $x$ is less than $a$, $|f(x)-g(a)|<\epsilon$. Since the difference is less than any positive number, $f(a)$ must equal $g(a)$.
\newline
Now consider the fact that  $\lim_{x \to a} \frac{h(x)-h(a)}{x-a}$ exists. Then we must have
$$\lim_{x \to a^+} \frac{h(x)-h(a)}{x-a} = \lim_{x \to a^-} \frac{h(x)-h(a)}{x-a}$$
By the definition of $h$ we can now break the above statement down into:
$$\lim_{x \to a^+} \frac{f(x)-f(a)}{x-a} = \lim_{x \to a^-} \frac{g(x)-g(a)}{x-a}$$
Since $f'(a), g'(a)$ exist, then the above equation must imply $\lim_{x \to a^+} f'(a) = \lim_{x \to a^-} g'(a)$, thus $f'(a)=g'(a)$


\subsection{Converse}
Let $f(a)=g(a), f'(a)=g'(a)$. Since $f(a)=g(a)$, and $f,g$ are differentiable and therefore continuous, the function $h$ is continuous at $a$.
Now consider the derivative: $\lim_{x \to a} \frac{h(x)-h(a)}{x-a}$.
\newline
We break it down into two cases: $x$ converging from the negative and the positive
$$\lim_{x \to a^+} \frac{h(x)-h(a)}{x-a} = \lim_{x \to a^+} \frac{f(x)-f(a)}{x-a} = f'(a)$$
$$\lim_{x \to a^-} \frac{h(x)-h(a)}{x-a} = \lim_{x \to a^-} \frac{g(x)-g(a)}{x-a} = g'(a)$$
Now since $f'(a)=g'(a)$, we know that $\lim_{x \to a^+} \frac{h(x)-h(a)}{x-a} = \lim_{x \to a^-} \frac{h(x)-h(a)}{x-a}$.
Thus the limit exists, and therefore $h$ is differentiable at $a$
\newline
$\blacksquare$


\end{document}
