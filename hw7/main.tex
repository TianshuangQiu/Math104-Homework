\documentclass[12pt]{article}
\usepackage[usenames]{color} %used for font color
\usepackage{amsmath, amssymb, amsthm}
\usepackage{wasysym}
\usepackage[utf8]{inputenc} %useful to type directly diacritic characters
\usepackage{graphicx}
\usepackage{caption}
\usepackage{subcaption}
\usepackage{float}
\usepackage{mathtools}
\usepackage [english]{babel}
\usepackage [autostyle, english = american]{csquotes}
\MakeOuterQuote{"}
\graphicspath{ {./} }
\newcommand{\Z}{\mathbb{Z}}
\newcommand{\N}{\mathbb{N}}
\newcommand{\R}{\mathbb{R}}
\newcommand{\Q}{\mathbb{Q}}
\newcommand{\prob}{\mathbb{P}}
\newcommand{\degrees}{^{\circ}}
\DeclarePairedDelimiter\ceil{\lceil}{\rceil}
\DeclarePairedDelimiter\floor{\lfloor}{\rfloor}

\author{Tianshuang (Ethan) Qiu}
\begin{document}
\title{Math 104, HW7}
\maketitle
\newpage


\section{Q1}
We first show that the limits exist $\implies$ uniformly continuous. We extend the definition of $f: [a,b]$ where $f(a) = \lim_{x\to a}, f(b) = \lim_{x\to b}$. Since the limit exists, we know that $f$ can get arbitrily close it. More precisely, we know that there exists $\delta$ such that $|x-a| < \delta \implies |f(x)-\lim_{x\to a}|<\epsilon$ for any $\epsilon > 0$. This also demonstrates continuity at $a$. We can repeat this logic to show that it is continuous at $b$. Then, since $f$ is continuous on the closed inverval and continuous, it is uniformly compact.
\newline
Now we show that uniformly continuous $\implies$ limits exist. Let $a_n \in (a,b)$ be an arbitrary sequence that converges to $a$. Then since it is convergent, we know that it is cauchy. Now since $f$ is uniformly continuous, we know that $f(a_n)$ is also cauchy, and it is therefore convergent. Therefore the limit exists as $x \to a$. We can repeat this argument for $b$ to show that $\lim_{x\to b} f(x)$ exists.
\newline
Thus it is proven.
\newpage


\section{Q2}
\paragraph{Forward statement}
\newline
Let $f: S \to S^{*}$ be continuous, let $E \subset S^{*}$ be a closed set, then $E' = S^{*} \setminus E$ is open and $f^{-1}(E')$ is also open by continuity.
\newline
Consider $f^{-1}(E')$, since $f$ is defined on all $S$, each $\exists f(s) \forall s \in S$. Therefore all $s \in S$ either has an image in $E$ or in $E'$, and $f^{-1}(E') = S \setminus f^{-1}(E)$. Since we know that both $S$ and $f^{-1}(E')$ are open, $f^{-1}(E)$ must be closed.

\paragraph{Converse}
\newline
Now assume that for any closed $F \subset S^{*}$,  $f^{-1}(F)$ is also closed. We essentially construct the same proof for open sets but with one more step. For any point $s_0 \in S$, $\epsilon >0$,  construct open set $G = \{d^{*}(f(s_0), s^{*}) < \epsilon\}$. Now take its complement $G' = S^{*} \setminus G$. Since $S^{*}$ and $G$ are both open, $G'$ is closed.
\newline
By our assuption $H' = f^{-1}(G')$ is closed, then its complement $H = S \setminus H'$ is open. Since $f(s_0) \in G$, we have $s_0 \in H$. Then we can find a "ball" inside this set such that for some $\delta >0$, $\{d(s-s_0)<\delta\} \subseteq H$
\newline
Thus it follows that $d(s-s_0)<\delta$ implies $d^{*}(f(s_0), s^{*}) < \epsilon$ and we have shown that $f$ is continuous at $s_0$. Since $s_0$ is an arbitrary point in $S$, $f$ is continuous.
\newpage


\section{Q3}
Let $C = \{(1/n, \infty)...\}$ for all $n \in \N$.
We claim that $C$ is a cover for $(0, \infty)$. To see that it is true, let $x \in (0, \infty)$, if $x>1$, it is in every subset. Otherwise, by the Archimedean Principle there exists $m \in \N$ such that $1/m < x$. Therefore $x \in (1/m, \infty) \in C$.
\newline
Now we take $C \cup (-\infty, 0]$. Let $x \in \R$, if $ x\leq 0$, then it is in the second set, otherwise it is in the first.
\newline
Let $C'$ be a finite subset of $C$, then we must have $C' = \{(1/a, \infty), ..., (1/b, \infty)\}$ with natural numbers $a \leq b$. Now consider $r = 1/(2a)$. Since $0<r<1/a$, $r \not \in C'$, and $r \not \in (-\infty, 0]$
\newline
We can find a such $r$ for any finite subset of $C$, therefore there is no finite subcover of $\R$. Thus $\R$ is not compact.
\newpage


\section{Q4}
Consider $X = S \setminus F$. Since $F$ is closed, its complement $X$ is open. Now let $C$ be any arbitrary open cover of $F$. We take the union of the above two sets: $C' = X \cup C$. Since $C$ covers F and $A$ covers the rest of the metric space, and since $E \subset S$, $C'$ is an open cover of $E$.
\newline
Now we apply the definition of compactness, so there exists a finite subcover in $C'$. Let this subcover be $Y$. If $X \not\in Y$, then our proof is complete since $F \subseteq E$. Any cover that covers $E$ must also cover $F$.
\newline
Otherwise, we remove A: $Y \setminus X$. Since we have removed what we have added, $(Y \setminus X) \subseteq C$. This set is a subcover for $F$, since $X \cup F = \emptyset$, we are not removing any point that is inside $F$, thus we have found a subcover that covers $F$.
\newpage


\section{Q5}

\end{document}
